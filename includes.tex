%%%%%%%%%%%%%%%%%%%%%%%%%%%%%%%%%%%%%%%%%
% University Assignment Title Page 
% LaTeX Template
% Version 1.0 (27/12/12)
%
% This template has been downloaded from:
% http://www.LaTeXTemplates.com
%
% Original author:
% WikiBooks (http://en.wikibooks.org/wiki/LaTeX/Title_Creation)
%
% License:
% CC BY-NC-SA 3.0 (http://creativecommons.org/licenses/by-nc-sa/3.0/)
% 
% Instructions for using this template:
% This title page is capable of being compiled as is. This is not useful for 
% including it in another document. To do this, you have two options: 
%
% 1) Copy/paste everything between \begin{document} and \end{document} 
% starting at \begin{titlepage} and paste this into another LaTeX file where you 
% want your title page.
% OR
% 2) Remove everything outside the \begin{titlepage} and \end{titlepage} and 
% move this file to the same directory as the LaTeX file you wish to add it to. 
% Then add \input{./title_page_1.tex} to your LaTeX file where you want your
% title page.
%
%----------------------------------------------------------------------------------------
%	PACKAGES AND OTHER DOCUMENT CONFIGURATIONS
%----------------------------------------------------------------------------------------
\usepackage{ifxetex}
% \usepackage{textpos}
% \usepackage{natbib}
\usepackage{kpfonts}
\usepackage[letterpaper,hmargin=2.8cm,vmargin=2.0cm,includeheadfoot]{geometry}
% \usepackage{ifxetex}
\usepackage{stackengine}
\usepackage{tabularx,longtable,multirow,subfigure,caption}%hangcaption
\usepackage{fncylab} %formatting of labels
\usepackage{fancyhdr}
\usepackage{color}
% \usepackage[tight,ugly]{units}
\usepackage{url}
\usepackage{float}
\usepackage[english]{babel}
\usepackage{amsmath}
\usepackage{graphicx}
\usepackage[colorinlistoftodos]{todonotes}
\usepackage{dsfont}
\usepackage{epstopdf} % automatically replace .eps with .pdf in graphics
% \usepackage{natbib}
% \usepackage{backref}
\usepackage{array}
\usepackage{latexsym}
\usepackage{etoolbox}

\usepackage{enumerate} % for numbering with [a)] format 

\usepackage[backend=biber,style=apa,citestyle=authoryear]{biblatex}
\addbibresource{solar.bib}

\ifxetex
\usepackage{fontspec}
\usepackage{tabularx}
\setmainfont[Scale=.8]{OpenDyslexic-Regular}
\else

% \hypersetup{pdftitle={},
%   pdfsubject={}, 
%   pdfauthor={\reportauthorOne},
%   pdfkeywords={}, 
%   pdfstartview=FitH,
%   pdfpagemode={UseOutlines},% None, FullScreen, UseOutlines
%   bookmarksnumbered=true, bookmarksopen=true, colorlinks,
%     citecolor=black,%
%     filecolor=black,%
%     linkcolor=black,%
%     urlcolor=black}


\usepackage{tcolorbox}

% various theorems
\usepackage{ntheorem}
\theoremstyle{break}
\newtheorem{lemma}{Lemma}
\newtheorem{theorem}{Theorem}
\newtheorem{remark}{Remark}
\newtheorem{definition}{Definition}
\newtheorem{proof}{Proof}

% example-environment
\newenvironment{example}[1][]
{ 
\vspace{4mm}
\noindent\makebox[\linewidth]{\rule{\hsize}{1.5pt}}
\textbf{Example #1}\\
}
{ 
\noindent\newline\makebox[\linewidth]{\rule{\hsize}{1.0pt}}
}



%\renewcommand{\rmdefault}{pplx} % Palatino
% \renewcommand{\rmdefault}{put} % Utopia

\ifxetex
\else
\renewcommand*{\rmdefault}{bch} % Charter
\renewcommand*{\ttdefault}{cmtt} % Computer Modern Typewriter
%\renewcommand*{\rmdefault}{phv} % Helvetica
%\renewcommand*{\rmdefault}{iwona} % Avant Garde


% \setlength{\parindent}{0em}  % indentation of paragraph

\setlength{\headheight}{14.5pt}
\pagestyle{fancy}
\fancyfoot[ER,OL]{\thepage}%Page no. in the left on
                                %odd pages and on right on even pages
\fancyfoot[OC,EC]{\sffamily }
\renewcommand{\headrulewidth}{0.1pt}
\renewcommand{\footrulewidth}{0.1pt}
\captionsetup{margin=10pt,font=small,labelfont=bf}


%--- chapter heading

\def\@makechapterhead#1{%
  \vspace*{10\p@}%
  {\parindent \z@ \raggedright %\sffamily
        %{\Large \MakeUppercase{\@chapapp} \space \thechapter}
        %\\
        %\hrulefill
        %\par\nobreak
        %\vskip 10\p@
    \interlinepenalty\@M
    \Huge \bfseries 
    \thechapter \space\space #1\par\nobreak
    \vskip 30\p@
  }}

%---chapter heading for \chapter*  
\def\@makeschapterhead#1{%
  \vspace*{10\p@}%
  {\parindent \z@ \raggedright
    \sffamily
    \interlinepenalty\@M
    \Huge \bfseries  
    #1\par\nobreak
    \vskip 30\p@
  }}
  

\usepackage{setspace}

\usepackage[colorlinks=true, allcolors=blue]{hyperref} % provide links in pdf

%%% Local Variables: 
%%% mode: latex
%%% TeX-master: "notes"
%%% End: 
\usepackage[all]{hypcap}