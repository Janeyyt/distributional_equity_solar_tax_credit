%%%%%%%%%%%%%%%%%%%%%%%%%%%%%%%%%%%%%%%%%
% University Assignment Title Page 
% LaTeX Template
% Version 1.0 (27/12/12)
%
% This template has been downloaded from:
% http://www.LaTeXTemplates.com
%
% Original author:
% WikiBooks (http://en.wikibooks.org/wiki/LaTeX/Title_Creation)
%
% License:
% CC BY-NC-SA 3.0 (http://creativecommons.org/licenses/by-nc-sa/3.0/)
% 
% Instructions for using this template:
% This title page is capable of being compiled as is. This is not useful for 
% including it in another document. To do this, you have two options: 
%
% 1) Copy/paste everything between \begin{document} and \end{document} 
% starting at \begin{titlepage} and paste this into another LaTeX file where you 
% want your title page.
% OR
% 2) Remove everything outside the \begin{titlepage} and \end{titlepage} and 
% move this file to the same directory as the LaTeX file you wish to add it to. 
% Then add \input{./title_page_1.tex} to your LaTeX file where you want your
% title page.
%
%----------------------------------------------------------------------------------------
%	PACKAGES AND OTHER DOCUMENT CONFIGURATIONS
%----------------------------------------------------------------------------------------
\usepackage{ifxetex}

\usepackage{kpfonts}
\usepackage[letterpaper,hmargin=2.8cm,vmargin=2.0cm,includeheadfoot]{geometry}
\usepackage{tabularx,longtable,multirow,subfigure,caption}%hangcaption
\usepackage{fncylab} %formatting of labels
\usepackage{fancyhdr}
\usepackage{color}
\usepackage{url}
\usepackage{booktabs}
\usepackage[english]{babel}
\usepackage{amsmath}
\usepackage{graphicx}
\usepackage[colorinlistoftodos]{todonotes}
\usepackage{dsfont}
\usepackage{array}
\usepackage{latexsym}
\usepackage{etoolbox}
\usepackage{enumerate} 
\usepackage[backend=biber,style=apa,citestyle=authoryear]{biblatex}
\addbibresource{solar.bib}
\addbibresource{solar_equity.bib}


% various theorems
\usepackage{ntheorem}
\theoremstyle{break}
\newtheorem{lemma}{Lemma}
\newtheorem{theorem}{Theorem}
\newtheorem{remark}{Remark}
\newtheorem{definition}{Definition}
\newtheorem{proof}{Proof}


\ifxetex
\else
\renewcommand*{\rmdefault}{bch} % Charter
\renewcommand*{\ttdefault}{cmtt} % Computer Modern Typewriter
%\renewcommand*{\rmdefault}{phv} % Helvetica
%\renewcommand*{\rmdefault}{iwona} % Avant Garde
\fi

% \setlength{\parindent}{0em}  % indentation of paragraph

% Set up headers
\fancyhead[RO]{\leftmark}
\fancyhead[LO]{ }
\fancyhead[LE]{ }
\fancyhead[RE]{\leftmark}



\setlength{\headheight}{14.5pt}
\pagestyle{fancy}
\fancyfoot[ER,OR]{\thepage}%Page no. in the left on
                                %odd pages and on right on even pages
\fancyfoot[OC,EC]{\sffamily }
\renewcommand{\headrulewidth}{0.1pt}
\renewcommand{\footrulewidth}{0.1pt}
\captionsetup{margin=10pt,font=small,labelfont=bf}

% Custom command to set a different section name in the header
\newcommand{\customsection}[2]{
  \section[#1]{#2}
  \markright{#1}
}


% %--- chapter heading

% \def\@makechapterhead#1{%
%   \vspace*{10\p@}%
%   {\parindent \z@ \raggedright %\sffamily
%         %{\Large \MakeUppercase{\@chapapp} \space \thechapter}
%         %\\
%         %\hrulefill
%         %\par\nobreak
%         %\vskip 10\p@
%     \interlinepenalty\@M
%     \Huge \bfseries 
%     \thechapter \space\space #1\par\nobreak
%     \vskip 30\p@
%   }}

% %---chapter heading for \chapter*  
% \def\@makeschapterhead#1{%
%   \vspace*{10\p@}%
%   {\parindent \z@ \raggedright
%     \sffamily
%     \interlinepenalty\@M
%     \Huge \bfseries  
%     #1\par\nobreak
%     \vskip 30\p@
%   }}
  

\usepackage{setspace}
\usepackage[colorlinks=true, allcolors=blue]{hyperref} 

\usepackage[all]{hypcap}

\usepackage{float}